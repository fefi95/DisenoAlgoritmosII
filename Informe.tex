\documentclass{ci5652}
\usepackage{graphicx,amssymb,amsmath}
\usepackage[utf8]{inputenc}
\usepackage[spanish]{babel}
\usepackage{hyperref}
\usepackage{subfigure}
\usepackage{paralist}
\usepackage[ruled,vlined,linesnumbered]{algorithm2e}

%-------------------------- Macros and Definitions ----------------------------%

% Add all additional macros here, do NOT include any additional files.

% The environments theorem (Theorem), invar (Invariant), lemma (Lemma),
% cor (Corollary), obs (Observation), conj (Conjecture), and prop
% (Proposition) are already defined in the ci5652.cls file.

%--------------------------- Title -------------------------------------------%

\title{Metaheurísticas para el problema de Aprendizaje de Pesos en Características}

\author{Stefani Castellanos
        \and
        Erick Silva}

%--------------------------------- Text --------------------------------------%

\begin{document}
\thispagestyle{empty}
\maketitle

\begin{abstract}
*Inserte una descripción breve del paper.*
\end{abstract}

\section*{Introducción}
*Inserte introducción*

\section{Descripción del problema}
*Inserte descripción*

\subsection{Representación}
*Inserte representación*

\subsection{Función objetivo}
*Inserte Función objetivo*

\section{Implementación}
*Inserte Implementación*

\section{Resultados}
*Inserte Resultados*

\section*{Conclusiones}

Aquí concluyen.

%------------------------------ Bibliography ---------------------------------%

% Please add the contents of the .bbl file that you generate,  or add bibitem entries manually if you like.
% The entries should be in alphabetical order
\small
\bibliographystyle{abbrv}

\begin{thebibliography}{99}

\bibitem{so2005}
C. So and H. So.
\newblock A groundbreaking result.
\newblock {\em Journal of Everything}, 59(2):23--37, 2005.

\end{thebibliography}


\newpage
\section*{Apéndice}

Bla.

%------------------------ EL EJEMPLO DE LA PROFE -----------------------------%

\section{La sección de ejemplo de la profe}
Citan así~\cite{so2005}. ELIMINAR AL ESCRIBIR EL INFORME

\begin{algorithm}
 \DontPrintSemicolon
 \vspace*{0.1cm}
 \KwIn{Descripcion}
 \KwOut{Descripcion}
 Primer paso\;
 Segundo\;
 \ForEach{$i = 1\dots n$}{
  \If{Alguna condición}{
   Algo aqui\;
   }
 }
 \KwRet{Valor}
 \vspace*{0.1cm}
 \caption{Nombre}
\end{algorithm}

\end{document}
